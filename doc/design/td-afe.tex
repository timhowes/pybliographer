% $Id$

\section{Features of commercial products}
\label{sec:commfeat}

\subsection{Biblioscape}
\label{sec:biblioscapefeat}


                                                               
%\textbf{Biblioscape 4 Feature Matrix}


\textit{Organize references with folders, dynamic folders, etc.}
\begin{itemize}
 \item[Folder] \label{biblioscapefolder}
Add references into folders. One reference can be put into
 multiple folders without creating a duplicate. Organize references
 into folders with drag and drop.

 
 \item[Dynamic folder] Organize and save queries into a tree structure.
 All references meeting the search criteria will be listed under a
 dynamic folder.

 \item[Indexed search] Return search results in a couple of seconds no
 matter how big the database is. One line search works the same
 way as most Internet search engines. Supports logical searches,
 fuzzy searches, etc.

 \item[Import filter] Bibliographic data from any data sources can be
 imported with a proper import filter. User can create new or edit
 existing import filters.

 \item[Output style] References can be displayed any any style like MLA,
 APA, etc. A large number of styles are provided for different
 journals. Users can also create new ones.

 \item[Cross linking] Link a reference to other references in the same
 database. You can define a relationship for the links, like
 "Supportive", "Contradict". You can also add comments for each
 link.

 \item[Navigation view] A reference can be displayed in an organizational
 chart where each node can lead to related records.

 \item[Formatted preview] Display a reference as formatted text
 according to the active output style. 

 \item[Live preview] Display data fields of the selected reference in a grid
 without opening it. Changes made to the data will be saved to
 database as you move to another record.

 \item[Graphics and OLE] If a reference has associated graphics and OLE
 objects, add them to the Document field. The document field can
 be used to store the full text of a reference.

 \item[Field lookup] List all unique values along with the number of
 occurrences of a data field. All data fields with possible repeated
 values can be shown in lookup view. These include Author,
 Keyword, Publisher, Language, Country, Subject, etc.

 \item[Recycle bin] All deleted references are put into the Recycle bin.
 You can recover them from the Recycle bin or remove them
 permanently from the Recycle bin.

 \item[Advanced search] Query any data field with a visual query builder.

 \item[Find and Replace] Search for a word or phrase and limit the search
 to a data field or all data. The same is true for the Replace
 operation.

 \item[Sorting] Sort a column by clicking on the header. Click again and
 the column will sort in reverse order. User can also define
 multi-level sorting.

 \item[Filtering] Define a filtering criteria with a visual filter builder and
 apply the filter to any dataset.

 \item[Term list] Users can keep frequently used phrases in a term list.
 Terms can be organized in the list by category.

 \item[Move field] The content of a data field can be moved from one
 field to another.

 \item[Global edit] The content of a data field can be changed at once
 for all selected references.

 \item[Eliminate duplicates] Duplicate records can be found and
 removed. Fuzzy search is supported for finding duplicates.

 \item[Analyze references] Data fields in the reference table can be
 analyzed for data distribution.

 \item[SQL commands] Users who are familiar with SQL can query the
 database directly with SQL commands.

 \item[Report] A built-in database report writer will a print data report,
 including a subject bibliography grouped by keyword, author, year,
 subject, etc.

\end{itemize}
\textit{Format papers to generate citations and bibliography}

\begin{itemize}
\item  [Format a paper] Convert the temporary citations of a document
 into formatted citations and a bibliography.

 \item[Unformat a paper] Convert a Biblioscape formatted paper back to
 unformatted form (with temporary citations) so citations can be
 added or deleted before the final formatting.

 \item[Word support] Full integration with Microsoft Word, Biblioscape
 menus and toolbar can be added to the Word menu and toolbar
 system.

 \item[WordPerfect support] Full integration with Corel WordPerfect,
 Biblioscape menus and toolbar can be added to the WordPerfect
 menu and toolbar system.

 \item[Other word processors] Biblioscape methods for word processors
 integration are published and open to all word processors that
 support DDE.

 \item[HTML support] Biblioscape can generate formatted papers in HTML
 format. A hyperlink can be created automatically between an
 in-text citation and its reference in a bibliography.

 \item[Natural citation] Use words or phrases to uniquely identify a
 reference in a temporary citation instead of using a Reference ID.
 If references are moved into another database, temporary citations
 don't need to be changed.

 \item[Cite while you write] Use BiblioSidekick to display references in a
  small, always on top windows. While in a word processor like Word
 or WordPerfect, just drag and drop the selected reference in the
 place where you want to cite it.

 \item[BiblioWord] A full featured word processor inside Biblioscape. Just
 drag selected references from a panel on the right when you want
 to cite. BiblioWord supports live spelling check, thesaurus, tables,
 graphics, OLE, multi-level undo, etc.

\end{itemize}

\textit{Access the Internet to capture bibliographic data, Web pages}
\begin{itemize}

 \item[Remote databases] Access thousands of remote bibliographic
 databases on the Web with an integrated Web browser. These
 sites include university sites, commercial databases, and
 government sites. Most of them are free.

 \item[Capture references] Search web based bibliographic databases
 from inside Biblioscape, click a button to capture search results
 into a Biblioscape database with the right import filter. New import
 filters can be created by users.

 \item[Capture Web pages] Research on the Web with the Biblioscape
 integrated browser, capture a web page into a Biblioscape
 References table or Notes table. All words in the Web page will be
 indexed for future search. Graphics and links are captured along
 with the page.

 \item[Resources] A directory of bibliographic resources on the Web. Each
 entry listed has an associated import filter. The local Resources list
 can be expanded and edited by the user.

 \item[Web directory] Biblioscape Web site lists a collection of sites
 valuable to researchers. Web sites are organized by subject.
 Bibliographic databases are the main part of the listing. Although
 other types of Web resources are also listed.

 \item[Z39.50] Most Z39.50 enabled bibliographic databases also have a
 Web interface, Biblioscape's integrated Web browser can be used
 to search such sites and capture search results directly into a
 database. 

 \item[Link to a note] Easily create a link between a note and a Web site.

\end{itemize}


\textit{Take notes and link them to references, tasks, web sites, etc.}

\begin{itemize}
 \item[Tree structure] Organize notes in a tree structure. Note's position
 in the tree can be rearranged by drag and drop.

 \item[Indexed search] Find your note fast with indexed search. Each
 word in your Notes database is indexed for super fast search. The
 search words are colored in red on the hit page. Indexed search
 supports logical operators, wildcards, fuzzy search, etc.

 \item[Advanced search] Limit your search to a data field like Date
 Created, Keywords, etc. Build complex searches with a visual query
 builder.

 \item[Format text] The text in your note can be formatted with all the
 standard options, including fonts, color, background color,
 superscript, subscript, paragraph alignment, bullet list, number list,
 etc.

 \item[Link] Each note can be linked to other notes, references, tasks,
 catalog items, Web URLs, local files, etc. Double clicking on a link
 will take you to the linked item.

 \item[Web capture] Notes can be used to organize captured Web pages.
  All the graphics and hyperlinks of captured web pages can be
 properly displayed.

 \item[Table support] You can insert tables in your notes. Additional rows
 can be added and deleted.

 \item[Find and Replace] Standard Find and Replace tools for finding and
 replacing text in your notes.

 \item[Graphics and OLE] Graphics can be added to your notes. OLE is
 also supported. Therefore, you can add chemical structure
 drawings, spreadsheets, CAD drawings, etc. in your notes.

 \item[Table view] The notes can also be displayed in a table besides the
 default tree view. Notes can be sorted and grouped in a table.

 \item[Keyword lookup] Each note can have associated keywords. These
 keywords can be displayed in a lookup list along with its number of
 occurrences. Double clicking on a keyword will retrieve all related
 notes.

 \item[Spelling and thesaurus] A powerful spelling checker is included.
 Additional dictionaries can be downloaded for all major European
 languages. A thesaurus is also included to help the user to find the
 right words during writing.

 \item[Icons] Each note can be assigned a different icon to distinguish it
 from other notes.

 \item[Export] Each note can be exported to a file in RTF or HTML format.

\end{itemize}

\textit{Manage tasks and organize your research ToDo list}
\begin{itemize}

 \item[Sort tasks:]Click on the column header to sort tasks, click again to
 sort in reverse order.

\item[Group tasks] Group tasks by Priority, Status, Date Created, etc.
]
 \item[Task progress] Track the progress of a task by marking its
 percentage completed.

 \item[Task creation] Create tasks inside References module, and add
 selected references into the Description field of the new task.

 \item[Link to a note] Create a link between a selected task and a note.

 \item[Advanced search] Search tasks with a visual query builder.

\end{itemize}

\textit{Draw a chart to present your ideas}
\begin{itemize}

 \item[Flow chart] Draw a flow chart with an easy to use chart editor.

 \item[Knowledge map] Draw a chart and link a chart object to other
 modules. For example, double clicking on a chart object will open a
 group of references, tasks, notes, etc. A SQL query can be
 associated with each chart object. A knowledge map can be built
 with such associated queries.

 \item[Tree structure] Organize your charts in a tree structure. The
 position of each chart in the tree can be rearranged by drag and
 drop.

 \item[Link to a note] Create a link between a chart and a note.

 \item[Zoom Display] options like zoom in and zoom out, actual size, and
 fit to screen are supported.

 \item[Icon] Each chart can have an icon associated and displayed.

 \item[Shape and color] The shape and color of each chart object can be
 customized. The label text can be displayed in different fonts and
 colors.

 \item[Connectors] Chart objects can be connected with a flexible
 connector which can be curved. A connector can have its own
 label, font, color, size, different sources and destination arrows,
 and link points.
 
\end{itemize}


\textit{Manage a library without a steep learning curve}
\begin{itemize}
 \item[Catalog] Manage library collection data into 56 data fields,
 organized into several groups including Bibliographic, Holding,
 Request, Order, Serial, and General.

 \item[Serials] Manage serials and related activities including tracking,
 routing, etc. 

 \item[Circulation history] Search, sort, and group circulation data.
 Display circulation activities by borrower, status, subject, etc. 

 \item[Check Out] Check out books for library patrons, add notes, easily
 change due dates.

 \item[Check In] Check in books returned by borrowers. Automatically
 reminds librarian about Hold status.

 \item[Renew] Renew books for borrowers, add a note. Find renewed
 items by ID or title.

 \item[Hold] Put a hold on a checked out book. Show a reminder when
 that book is returned.

 \item[Interlibrary Loan] Manage interlibrary loan requests, track loan
 status, log shippings, etc.

 \item[Borrowers] Manage borrower's information (address, phone, fax,
 email, et]c.)

 \item[Lenders] Manage lender's information (contact's name, phone, fax,
 email, notes, etc.)

 \item[Suppliers] Manage supplier's information (address, phone, fax,
 email, notes, etc.)

 \item[Sort] Click on any column header to sort then click again to sort in
 reverse order.

 \item[Group] Group data by drag and drop. Data can be grouped at
 multi-levels by any data field.

 \item[Field chooser] Choose which data fields to include in the data grid
 by drag and drop.

 \item[Report and print] Build or customize data reports with a powerful
 report builder. Users can create new reports with a wizard. New
 reports can be easily added to the menu system. Reports can be
 previewed, printed, or saved as a files.


\end{itemize}

\textit{Web enable your bibliographic database with one click}

\begin{itemize}

 \item[Web publishing] Publish databases on the Web with BiblioWeb
 server. No other web server required. Runs on any Windows 95, 98,
 Me, NT4, 2000 machine.

 \item[Indexed search] Search references with a powerful search engine.
 Enter search commands like you do with a Web search engine.
 Supports search keywords AND, NOT, OR, LIKE, NEAR, Wildcards,
 etc.

 \item[Advanced search] Limit searches to certain fields. Build complex
 queries with up to 3 conditions.

 \item[Add references] Users with a Write account can add new
 references to the database using a web browser.

 \item[Edit and delete] Users can edit or delete their own references over
 the web.

 \item[Import] Import references over the Web with the right import filter,
 so you don't need to enter references one by one.

 \item[Hyperlinks] Search results are displayed with hyperlinks. Clicking
 on the hyperlink will trigger a new search for related items.

 \item[Style] Marked references can be displayed in any of the output
 styles that exist in Biblioscape.

 \item[Export] Marked references can be exported in several formats to be
 easily imported into other programs.

 \item[Format papers] Users can even format a paper over the Internet.
 Temporary citations in a document will be converted to formatted
 citations and bibliographies.

 \item[User forum] Includes a user forum application, so you can host a
 web based forum without extra cost.


\end{itemize}

%%% Local Variables: 
%%% mode: latex
%%% TeX-master: "td-td1"
%%% End: 
