% $Id$

\section{Searching, Selecting, and Formatting Data}
\label{sec:processing}


A \textit{search} is an operation that is performed against one or
more \textit{databases} -- usually locally represented by
Pybliographer and remotely by database servers known as
\textit{connections} -- and that results uniformly in a \textit{result
  set}.

As a rule, such a result set is if necessary \textit{imported} as such
into the Pybliographer database, and sooner or later \textit{selected}
from.

A \textit{selection} is any subset of the database, usually the result
of manually inspecting (\textit{scanning}) some result set, that is
presented to a processing option: either to be set apart for further
processing at a later time, or to be processed by the output routines
in order to be presented or exported.
[XXX Are there other uses? ]



\paragraph{Notes:}

\begin{itemize}
\item Results sets can vary considerably in size. Efficiant means of
  handling them are important.
\item Various situations are to be considered and equally suppported
  \begin{itemize}
  \item One looks up price/holdings/availability information for an
    already known title.  That should result in the new information
    being added to the existing entry (and perhaps the latter being a
    bit improved). Possible actions include posting a note to pick up
    the book at the next visit to the library, or to make a copy from a
    journal, or to order it from the bookseller or the ILL service.
  \item One browses several data bases and cumulates the
    results. Later one reduces the set and adds it to the data base.
  \item One adds, perhaps on a regular schedule, records from an
    external source. Evidently this situation calls for other
    mechanisms and tools than the previous one, in spite of its
    superficial similarity. 
  \item On builds a bibliography or similar documentation database --
    involves most of the above situations and adds requirements for
    stringent administrative control. 
  \end{itemize}
\end{itemize}



\subsection{A generic query service}
\label{sec:genquery}

The exact possibilities that a database provides for searching are as
manifold as are the data. Paradoxically this makes it more easy to
provide one integrated query service -- any attempt to equip every
database and every service with its own query interface cannot but
falter in view of the enormeous programming investment that would be
needed.

It's not clear, perhaps, how the individual services can communicate
their capabilities upwards towards the query service, but it should be
evident that \textit{some} communication would be helpful -- even if
one fully acknowledges that the assessment of a database service must
usually include information that is not available to the
program. (Example: cataloguing rules -- RAK, AACR, \dots)





\subsection{Integration with editors, wordprocessors, and browsers}
\label{sec:wpintegr}
\task{wpintegration}

Integration with editors like Emacs, word processors like Abiword, and
bowsers like Mozilla enhances ease of use, broadens its field of
application and reduces user's keystrokes and thus the opportunity for
errors on his side.

We distinguish the following use cases:
\begin{description}
\item[Adding a citation] When writing, the user wants to add a
  reference to certain document, of which he might remember no more
  than some words from the title, or the author, or even less, but
  as a rule, not the exact citation key that Bib\TeX\,e.g., would use
  to identify it.

  The user should refer to the document by selecting from a result set
  or shortlist. 
\item[Inserting citations] Quite often the case arises that a document
  must be cited in, say, an email without wanting the whole (Bib\TeX)
  machinery  to be put in action. 
  
  A simply formatted citation could be offered via \textit{Drag and
  Drop}.
\item[Annotating] When reading, the user wants to add a note. 
  
  He is helped with creating and maintaining annotations to documents
  (registering them with Pybliographer); annotations can be large and
  are not suitable for storage \textit{in} the bibliographical record,
  but a link could. On the other hand it is conceivable to point back
  from the annotation to the document, so it could stand on its own.
\item[Viewing documents] An electronic document could be requested.

  Define appropiate viewers and requestors (mainly for eprints -- the
  rest is more simple).
\item[Extracting metadata]  An electronic document is perused and the
  wish is that it be catalogued.
  
  A limited bibliographical description can be build from the metadata
  that is stored and communicated with the document.
  \textit{Importantly:} If downloaded, the local file address should
  be stored as well.
\end{description}
\subsubsection{Emacsen}


\subsubsection{Other editors}


\subsubsection{Mozilla}


\subsubsection{Other browsers}



\subsubsection{Kword}


\subsubsection{Abiword}


\subsubsection{Open Office}


\subsubsection{Other word processors}



\subsubsection{Other applications}



\subsection{Formatting references  and bibliographies}
\label{sec:formatting}


\subsection{Other applications}
\label{sec:applications}

[Connecting to OPACS, etc. ]

%%% Local Variables: 
%%% mode: latex
%%% TeX-master: "td-td1"
%%% End: 
