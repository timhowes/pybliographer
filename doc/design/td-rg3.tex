
\chapter{Preparing to Use Pybliographer}
\label{cha:rgprep}

This chapter discusses how to install and configure \Pyb.


It  describes how to do the following:

\begin{enumerate}
\item Install \Pyb.
\item Set up the required database structure
  \begin{itemize}
  \item Determine which databases to use
  \item Define the directory structure or access path
  \item Define additional data that you might need
  \end{itemize}
\item Set up \Pyb\ for easier operation
\item Load needed data
\item Write local processing routines
\end{enumerate}


% \chapter{Describing ressources}
% \label{cha:rgdesc}


% \section{General considerations}
% \label{sec:descgen}


% When describing ressources of any kind, we encounter some problems
% again and again. 

% The first has to do with the question which information to use; more
% specifically where to look for the information.

% To reduce the variation and confusion when describing ressources, there
% are certain established principles, among others:

% \begin{itemize}
  
% \item Use the original text, if possible. This principle avoids e.g.,
%   the confusion arising from giving a place of publication that is
%   \textit{K�ln, C�ln, Keulen, Cologne}, depending upon time and place
%   of cataloguing. 
% \item Use special normalised forms to account for the needs of users
%   that might not know the original form, but do not replace the
%   original form (for important pieces of information), but give both
%   of them.
% \item Prefer the information as it is found on the title page (and
%   some comparable places).  At least indicate information that does
%   not come from the standard places by putting it into brackets [].
% \item Distinguish the various classes of information, in particular do
%   not confound formal and material description, nor bibliogrpahic
%   description and annotations etc. although the software supports it
%   both. It much simplifies life to keep these purposes apart.
% \end{itemize}


% \section{Persons}
% \label{sec:rgpersons}




%%% Local Variables: 
%%% mode: latex
%%% TeX-master: "td-td2"
%%% End: 
