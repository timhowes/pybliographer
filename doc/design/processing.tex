% $Id$
% $Id$

\section{Searching, Selecting, and Formatting Data}
\label{sec:processing}


A \textit{search} is an operation that is performed against one or
more \textit{databases} -- usually locally represented by
Pybliographer and remotely by database servers known as
\textit{connections} -- and that results uniformly in a \textit{result
  set}.

As a rule, such a result set is if necessary \textit{imported} as such
into the Pybliographer database, and sooner or later \textit{selected}
from.

A \textit{selection} is any subset of the database, usually the result
of manually inspecting (\textit{scanning}) some result set, that is
presented to a processing option: either to be set apart for further
processing at a later time, or to be processed by the output routines
in order to be presented or exported.

[XXX Are there other uses? ]



\subsection{A general query service}
\label{sec:genquery}

The exact possibilities that a database provides for searching are as
manifold as are the data. Paradoxically this makes it more easy to
provide one integrated query service -- any attempt to equip every
database and every service with its own query interface cannot but
falter in view of the enormeous programming investment that would be
needed.

It's not clear, perhaps, how the individual services can communicate
their capabilities upwards towards the query service, but it should be
evident that \textit{some} communication would be helpful -- even if
one fully acknowledges that the assessment of a database service must
usually include information that is not available to the
program. (Example: cataloguing rules -- RAK, AACR, \dots)


\subsection{Integration with editors and wordprocessors}
\label{sec:wpintegr}
\task{wpintegration}


\subsection{Formatting references  and bibliographies}
\label{sec:formatting}



%%% Local Variables: 
%%% mode: latex
%%% TeX-master: "todo"
%%% End: 
