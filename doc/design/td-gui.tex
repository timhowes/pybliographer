\chapter{The Graphical User Interface}
\label{sec:gui}


In this chapter we discuss the user interface, in our case based on
the \textsf{Gtk\,2.0} toolkit. First we give an overview of the user
interface elements, then we look at the requirements of individual
widgets.

\textbf{See also:} For the integration of Pybliographer into Word
Processors and Browsers, see section \ref{sec:wpintegr}; for using
Pybliographer from the commandline or scripts, see chapter
\ref{sec:scripting}.


% \section{Major elements of the user interface}
% \label{sec:guimajor}

% \ref{sec:docrequest}


\section{Menus}
\label{sec:menus}\index{menu}

The menus are the main way to access \Pyb's functions, and must
accordingly be laid out with care. 
All functions  should  be accessbile through the menu interface.

Usually the main menu is laid out as follows:\\ \indent
 File \textbullet  Edit \textbullet View \textbullet \dots Help


This is an organisation that has been critised often enough, as being
appropiate only for one special case of program, viz.\ an editor like
application.  It is, however, well established.

Nevertheless, there is broad support for an adapted menu structure, if
the application doesn't fit the mold of an editor.

\subsection{The Menu Bar}
\label{sec:menusbar}


So I propose the following:
\begin{description}
\item[Pybliographer] The fulcrum of the application. Contains the
  \textsf{Preferences\dots} button, and then buttons for
  \textsf{Database}, \textsf{Folder}, \dots , each of which
  immediately switches  to the respective main view. In addition, exit
  function. 
\item[VarObject] Where Varobject is Database, \dots\ depending upon
  the preceding choice.

  
\item[Edit]  Contex dependent standard editing, 
\item[View]  Select view options, among them for window layout, e.g.,
  toolbat toggle, and for record display. \textit{Perhaps:} elementary
  formatting options for entries
\item[Go]  Necessary to navigate the result sets
\item[Search] To select the search dialog, to select a query for
  editing or execution, and to select an index for browsing.
\item[Format] 
\item[Scripts] to run external scripts (perhaps even to write them?)
  \textit{presuming that the user uses this way of customisation at all}
\item[Window] \textit{if there should be multiple windows}
\item[Help] Standard help menu
\end{description}



\subsection{The Application Menu}
\label{sec:guimnappl}

\begin{dnote}
\item  Compare also the Apple  Aqua Guidelines.
\end{dnote}


\subsection{The Database Menu}
\label{sec:guimndb}


\subsection{The Folder Menu}
\label{sec:guimnfold}


\subsection{The Record Menu}
\label{sec:guimnrecord}



\section{Displaying and Using Indices}
\label{sec:guiindex}


The usual way of perusing a bibliography or a catalogue is by means of
a listing, either according to the author's name, or according to the
title, a keyword, or any other suitable piece of information. 

With online catalogues, one usually has access to the underlying
(technical) \textit{indices}, if only to compensate for the
limitations of usual search facilities and the missing overview
possible with today's display devices. 

It is best to have multiple indices (Indexviews), so as to make best
use of the structure of the data (e.g., the \textit{Person or Author
  Index} would usually show the works of the author upon selection (as
a tree view) not the authority record of the author, a title index
might in a similar way allow subordinate entries for editions or
translations, etc. 

Sometimes special indices are required. A manuscript collection might
want to enter and find listed the first few words of a manuscript or
letter, a specialist might need the names of translators, or scribes,
or printers. So there must be a degree of configurability in addition
to that needed to cater for the differing tastes and traditions.

\subsection{Marking items}
\label{sec:marking}

A mark is available in every list view to the left of an entry, it is
used as the indicator for one special folder ``Marked''. 
By toggling the mark, one can easily change the exten



\subsection{Accelerator Key Assignments}
\label{sec:keys}

Accelerator keys allow easy selection of functions, in particular easy
switching of views, which is in particular desireable when doing data
entry and editing. These are activities which repeatedly and
predictably need a great number of screens to proceed, switching them
with the mouse is particularly unpleasant. 
Consitent assignment of \textit{shortcuts} is needed.



\cbstart

\section{Appendix: Accessability Guidelines}
\label{sec:guiaccess}

From \url{www.pro.gov.uk/recordsmanagement/2002referencefinal.pdf}

These samples guidelines are
indicative only, and are drawn from information available at
\url {www.state.me.us/CIO/accessibility/software_policy.html}


\begin{itemize*}
\item A program must provide keyboard access to all functions of the
  application. All actions required or available by the program must
  be available with keystrokes, i.e., keyboard equivalents for all
  mouse actions including but not limited to, buttons, scroll windows,
  text entry fields and pop-up menus.
\item A program must have a keyboard control sequence among all
  program controls and focal points. (e.g. using the tab key to
  navigate among edit fields, text boxes, buttons, and all other
  controls).
\item The focus must follow the keystroke, that is, using the arrow
  keys to navigate through a list followed by pressing the ENTER key
  or spacebar to select the desired item.
\item The software shall not interfere with existing accessibility
  features built into the operating system, such as Sticky keys, Slow
  Keys and Repeat Keys.
\item Timed responses are not to be used unless the timing parameter
  can be adjusted by an individual user.
\item There shall be selectable visual and auditory indication of key
  status for all toggle keys. (i.e. visual and auditory status
  indicators for keys such as the Number Lock, Shift/Caps Lock, and
  Scroll Lock keys.
\item All icons shall have clear precise text labels included on the
  focus or provide a user-selected option of text-only buttons.
\item The use of icons shall be consistent throughout the application.
  Pull-down menu equivalents must be provided for Icon functions
  (menu, tool and format bar).
\item There must be keyboard access to all pull-down menus.
\item For graphic text, system text drawing tools or other industry
  standard methods must be used so that screen reader software can
  interpret the image.
\item A visual cue for all audio alerts must be provided. The Sounds
  feature must be supported where built into the operating system. The
  user must be allowed to disable or adjust sound volume.
\item Colour-coding is not to be used as the only means of conveying
  information or indicating an action. An alternative or parallel
  method that can be used by individuals who do not possess the
  ability to identify colours must always be provided.
\item The application must support user defined color settings system
  wide. Highlighting should also be Viewable with inverted colors.
\item No patterned backgrounds behind text or important graphics are
  to be used.
\item User adjustment of, or user disabling of flashing, rotating or
  moving displays must be permitted to the extent that it does not
  interfere with the purpose of the application.  Consistently
  position the descriptions or labels for data fields immediately next
  to the field.
\item All reports and program output must be available in a format
  that is accessible by screen readers and other access systems.
\end{itemize*}

\cbend


%%% Local Variables: 
%%% mode: latex
%%% TeX-master: "DG"
%%% End: 
