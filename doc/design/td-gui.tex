\section{The Graphical User Interface}
\label{sec:gui}



\subsection{Displaying and Using Indices}
\label{sec:guiindex}


The usual way of perusing a bibliography or a catalogue is by means of
a listing, either according to the author's name, or according to the
title, a keyword, or any other suitable piece of information. 

With online catalogues, one usually has access to the underlying
(technical) \textit{indices}, if only to compensate for the
limitations of usual search facilities and the missing overview
possible with today's display devices. 

It is best to have multiple indices (Indexviews), so as to make best
use of the structure of the data (e.g., the \textit{Person or Author
  Index} would usually show the works of the author upon selection (as
a tree view) not the authority record of the author, a title index
might in a similar way allow subordinate entries for editions or
translations, etc. 

Sometimes special indices are required. A manuscript collection might
want to enter and find listed the first few words of a manuscript or
letter, a specialist might need the names of translators, or scribes,
or printers. So ther must be a degree of configurability in addition
to that needed to cater for the differing tastes and traditions.

\subsection{Accelerator Key Assignments}
\label{sec:keys}

Accelerator keys allow easy selection of functions, in particular easy
switching of views, which is in paricular desireable when doing data
entry and editing. These are activities which repeatedly and
predictably need a great number of screens to proceed, switching them
with the mouse is particularly unpleasant. 
Consitent assignment of \textit{shortcuts} is needed.



%%% Local Variables: 
%%% mode: latex
%%% TeX-master: "todo"
%%% End: 
