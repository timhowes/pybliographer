
\chapter{Pybliographer Concepts}
\label{cha:rgconc}

This chapter summarises some basic concepts that you need to
understand before you can use \Pyb.
It briefly describes:
\begin{itemize*}
\item What bibliographic items are and how they are described
\item What basic data types are used 
\item How the data is stored and processed
\item How \Pyb\ can help you storing and editing bibliographic
  resources and using them in you work
\end{itemize*}


\section{The Content of the Database}
\label{sec:rgccont}

The \Pyb\ database contains records of many types.  The following main
classes can be distinguished: 
\begin{itemize*}
\item Records that contain classifying information \textit{in the
    widest sense}, and which are loosely called \textsf{\itshape
    Topics}, as they are subclasses of \textsf{Topic}.
\item Records that contain the bibliographic description proper
\item Records that contain additional data, loosely called
  \textit{annotations}.
\item Internal records, containing \textit{configuration objects}.
\end{itemize*}



\subsection{Informations that Structure the Database}
\label{sec:rgconctopic}

The information that we speak about in this section represent a
concepts of bewildering variety -- however they all serve the common
purpose of organising the \textit{payload} of the database and share a
common technical base.

\begin{description}
\item[Classifiers: Languages, Genres, Forms, etc.] Without doubt, this
  is the least difficult kind of data used in connection with
  bibliography.  We have no taxonomic (tree-like) structure, but only a
  linear list of possible values, and mostly but little in way of name
  variants (\textit{see} references), only the need for translations of
  the list entries are a small complication that implies a little by
  way of programming.
  
  In most cases, such values are not \textit{indexed} in the sense
  that a list of items such as those that fulfil ``Language = XYZ'' is
  maintained by the system.  A query for all such items is then 
  necessarily done by scanning the whole database.  See
  \ref{sec:rginstcodes} for more informations.
\item[Keywords, Descriptors] Look similar but are very different
  because they do not come from a (mostly) pre-defined set.  That
  makes a difference not only for the programmer, but also for the
  user -- the longer list is not as easy to overview, see
  \ref{DG.sec:orgkeys} for a discussion of keywords.
  
  Open-ended lists occur also in connection with externally defined
  descriptors, such as the enzyme (EC) numbers. These are considered
  classifications (degenerate, if they do not form a tree, v. infra).
\item[Subject headings] on the other hand have a \textit{taxonomic}
  structure, they are linked to permit navigation to and from broader
  and narrower terms as well as related or opposite terms.  This comes
  at a price (no surprise) and most users will content themselves with
  what external databases provide.  Note, however, that \Pyb\ comes
  with a relatively sophisticated model of subject headings, which you
  could use to your profit by mapping incoming externally defined
  values to the preferred set, if you have one available for your area
  of work. 
\item[Classifications] The other generally used means of adding
  subject information to bibliographical records is by classifying the
  items described.  A classification, as e.g., the Mathematical
  Subject Classification \citep{MCS2000} comes as a tree where usually
  the leaves are eligible classes.  One typical property of
  classification is the use of codes (notations) to represent the
  classes, so is, e.g., \textsf{14G35} the code for ``Modular and
  Shimura varieties'' in the MCS with note ``[See also \textsf{11F41,11F46,
  11G18}]'' added. 
\item[Persons, Corporate Bodies] Persons or names of persons, and also
  corporate bodies, are well-recognised entities in library systems
  since long; this has not been the case for reference manager
  applications.  It is, however, obvious that they profit as well. 
\item[Works, Expressions] It has been usual, to varying degrees, to
  collect manifestations or expressions together which belong to one
  \textit{work}, so as to simplify the search and facilitate the
  selection.  It has proved even more necessary with on-line
  catalogues which are less easy to peruse.  A work is an optional
  element in that the average item has no need for it (it could be
  considered itself work and manifestation in one, without any
  consequences following from it) and under the control of the user,
  who groups the items, if the need arises.  This replaces
  functionally the entry of a uniform title.  
\end{description}\noindent

Missing in the above are user interface elements, like folders or
lists, that equally structure the database  and group items, but  that
are not equipped with the machinery, nor have the stability (and cost)
of the above mentioned tools.

Equally missing is the venerable concept of an entry type, like
\textit{article, proceeding} or even \textit{e-mail}.  It is replaced
by a far superior scheme, that separates concerns, allows for data
entry that is adapted to your needs, and does not arbitrarily restrict
you. 

\subsection{The Bibliographical Description}
\label{sec:rgconcbiblio}

Functionally the bibliographical records could be divided into a
the description proper and annotations, logically into a core and
extensions, technically into structured and unstructured, i.e.,
textual parts.

\begin{description}
\item[Core] A bibliographical record contains minimally some
  administrative data, and a title\footnote {A missing title is
    \textit{missing}, i.e., irregular, not \textit{not applicable}.}
  more precisely a `Title and statement of responsibility area'
  followed optionally by an `Edition area' (ISBD).
  
\item[Extended description] The old description was tilted towards
  books, this is nowadays but one of the possibilities. The format
  allows any number of extended descriptions, to accommodate all
  needs. 
\item[Associations] Authors, subject and similar informations are
  considered entities, thus they are not stored in the bibliographical
  record, but only a reference is.

\item[Annotations] are textual notes (but perhaps also structured
  \textit{case annotations}) including quotes from the item. -- This
  is a mixed-up class, that could need some further elucidation.


\item[Holdings] information refers to the actual location (incl. net
  and file system addresses) of an item; it can also include a
  reproduction specification.
\end{description}



\section{Interfaces}
\label{sec:rgcintf}

For (i) accessing and (ii) for extending. 

\section{Handling Associations}
\label{sec:rgcassoc}



% \section{The Peculiarities of Bibliographical Data}
% \label{sec:rgcdata}


% \begin{dnote}  
% \item Three types of data: Tabular, Textual, Complex, plus network of
% associations. Examples
% \end{dnote}


% \subsection{Tabular Data}
% \label{sec:rgconctab}

% Most data is stored and processed as the familiar tabular type, as a
% row of data fields, each with an associated \textit{domain} of
% permissible values. The program processes these data either as
% tuples\slash lists of values or as dictionaries with named fields.

% \begin{table}[htbp]\sffamily
%   \begin{tabular}[t]{|l|p{10cm}|}
% \hline \textbf{Type}& \textbf{Description}\\ \hline
% \textbf{Address}& Any address of publisher, author, user.\\
% \textbf{User}& User decription\\
% \textbf{HoldingT}& description of institution, filing  place
%                    etc. holding copy\\ 
% \textbf{Imprint ?}& Publication statement\\
% \textbf{Collation ?}& Extent information (for printed materials
%                      only?)\\ 
 
% \hline
%   \end{tabular}
%   \caption{Data in tabular form}
%   \label{tab:tabulardata}
% \end{table}



% \subsection{Complex Data}
% \label{sec:rgcoccompl}

% A minority of data elements is better handled through a higher level
% interface.  This \textit{complex data} comprises, hawever, the most
% prominent and important data elements of bibliographic records (no
% surprise, I suppose). 

% \begin{table}[htbp]\sffamily
%   \begin{tabular}[t]{|l|p{10cm}|}
% \hline \textbf{Type}& \textbf{Description}\\ \hline
% \textbf{Names}& Personal, corporate, and names of concepts\\
% \textbf{TitleEd}& Title and edition statement\\
% \textbf{NoteT}& Notes and annotation data like quotes\\
% \hline
%   \end{tabular}
%   \caption{Data in complex form}
%   \label{tab:complexdata}
% \end{table}





% \chapter{Working with Pybliographer}
% \label{cha:rg2}

% \newcommand{\Pyb}{Pybliographer{}}

% When you work with \Pyb, you will do so most often through one of the
% following windows:

% \begin{itemize}
% \item The Search Window
% \item The Folder Window
% \item The List Window
% \item The Detail Window
% \end{itemize}

% Often you will start with the Search Window. Enter a search term and
% start a search.  (See \ref{fig:searchx1}) The same window allows you to
% query another (possibly remote) database, and to formulate more
% complicated queries.

% Alternatively, you can open a predefined collection of records, a
% \textit{folder} or \textit{list}.  The \textit{Folder Window} allows
% you to select from all the folders and lists available. It can also be
% shown permanently as a left sidebar. Click on the small triangle to
% the left of each entry to open or close a folder in this display. Click
% on the name to display its contents in the list window.

% A list is simply a folder that is found under the label Lists at the
% beginnign of the folder display; together with the entries Marked,
% \dots

% Whatever your way, you will sooner or later see a \textit{List
%   Window}. It shows your current \textit{selection}. Usually records
% are ordered by name of first author and are displayed in a condensed
% format. Those are options that are easily set and changed; it is
% possible to tie thses settings to a particular list; think os a
% shopping list for you rnext visit to the library, that you might like
% always sorted by location.
  
% To inspect an item it is often sufficient to point with the mouse to
% it, and have a tooltip like preview window appear. It allows at the
% same time to concisely inspect  the entry and also to preview it in a
% typical format (both can be configured, of course).  Alternatively,
% you may open the deatil view (inspector) for an item. It presents you
% with a lot of information in a notebook widget. 







%%% Local Variables: 
%%% mode: latex
%%% TeX-master: "UG"
%%% End: 
